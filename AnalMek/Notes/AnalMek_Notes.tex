\documentclass[]{article}
\usepackage{preamble}
\usepackage[british]{babel}
\usepackage{amsthm}
\usepackage{bm}

\newtheorem{theorem}{Theorem}
\newtheorem{postulate}{Postulate}

\theoremstyle{definition}
\newtheorem{definition}{Definition}
\newtheorem{remark}{Remark}

%opening
\title{Notes for Analytical Mechanics}

\begin{document}

\maketitle

\section{Mathematics}
\begin{theorem}[Chain rule] If $F(q_1, \cdots, q_n)$ is a function of $n$ variables, then
	\begin{equation*}
		dF = \sum_i \dpt{F}{q_i} dq_i.
	\end{equation*} 
In particular, if $F(q_1, \cdots, q_n, t)$ is a function of coordinates and time, then
\begin{equation*}
	\dot F = \frac{dF}{dt} =  \sum_i \dpt{F}{q_i} \dot q_i + \dpt{F}{t}.
\end{equation*}
\end{theorem}

\begin{definition}[Levi-Cevita symbol]
	The Levi-Civita symbol $\epsilon_{i_1 \cdots i_n}$ is the sign of the permutation $({i_1 \cdots i_n})$.
	It has the following properties:
	\begin{enumerate}
		\item It is always $1$, $-1$ or $0$.
		\item Swapping any two indexes inverts the Levi-Civita symbol.
		\item $\epsilon_{i_1 \cdots i_n} = 0$ if $i_\alpha = i_\beta$ for any $\alpha \neq \beta$.
		\item A cyclic permutation does not change the Levi-Civita symbol.
	\end{enumerate}
\end{definition}

\section{Newtonian mechanics}
\begin{theorem}
	The work done by a force equals the change in kinetic energy:
	\begin{equation*}
		W_12 = \int_1^2 \mathbf{F} d\mathbf{s} = T_2 - T_1
	\end{equation*}
\end{theorem}

\begin{definition}[Conservative force]
	A \emph{conservative force} is the gradient of a scalar potential:
	\begin{equation*}
		\mathbf{F} = -\nabla V(\mathbf{r})
	\end{equation*}
\end{definition}

\begin{theorem}
	A conservative force conserves energy:
	\begin{equation*}
		T_2 - T_1 = W_{12} = V_1 - V_2
	\end{equation*}
\end{theorem}

\begin{definition}[Lagrange multipliers]
	content
\end{definition}

\section{Constraints, coordinates and displacements}
\begin{definition}[Holonomic constraint]
	Consider a system of $n$ particles with (actual, physical) coordinates $\mathbf{r_1}, \cdots \mathbf{r_n}$ and the time $t$. A \emph{holonomic} constraint on the system can be expressed as a set of equations
	\begin{equation*}
		f(\mathbf{r_1}, \cdots \mathbf{r_n}, t) = 0.
	\end{equation*}
\end{definition}

\begin{remark}
	We generally limit ourselves to holonomic constraints, and at a microscopic scale, these are generally the only ones involved.
\end{remark}

\begin{definition}[Rheonomous and scleronomous constraints]
	A constraint is called \emph{rheonomous} if it contains the time as an explicit variable. If it does \emph{not}, it is called \emph{scleronomous}.
\end{definition}

\begin{definition}[Virtual displacement]
	content
\end{definition}

\begin{postulate}[d'Alembert's principle]
	In the physical systems that we are interested in, reaction forces do no work under a virtual displacement.
\end{postulate}

\begin{definition}[Generalized force]
	\begin{equation*}
		Q_\sigma = \sum_{i=1}^{3N} F_i^{(a)} \dpt{x_i}{q_\sigma}
	\end{equation*}
\end{definition}

\section{Lagrangian mechanics}

\begin{definition}[Lagrangian]
	The Lagrangian is the kinetic energy $T$ minus the potential energy $V$:
\begin{equation*}
	L = T - V
\end{equation*}
\end{definition}

\begin{theorem}
	[Euler-Lagrange equation, without constraints]
	\begin{equation*}
		\dpt{L}{q} = \frac{d}{dt}\qty(\dpt{L}{\dot q})
	\end{equation*}
\end{theorem}

\begin{definition}[Action]
	The \emph{action} $S$ of a system evolving from $t_1$ to $t_2$ is the integral of the Lagrangian:
\end{definition}

\begin{equation*}
	S = \int_{t_1}^{t_2} L(t) \ dt
\end{equation*}

\begin{definition}[Variation]
	Variation:
	\begin{equation*}
		\delta S = 
	\end{equation*}
	\todo{what is this really?}
\end{definition}

\begin{postulate}[Hamilton's principle]
	A system will evolve in such a way that the action has a stationary value at the actual path of motion:
\begin{equation*}
	\delta S = 0
\end{equation*}
\end{postulate}

\section{Hamiltonian mechanics}
\begin{definition}
	[Canonical momentum from the Lagrangian]
\begin{equation*}
	p_i = \dpt{L}{\dot q_i}
\end{equation*}
\end{definition}
\begin{definition}[Hamiltonian]
	The Hamiltonian ($i$ running over all variables) is calculated from a Lagrangian $L(\qty{q_i, \dot q_i}, t)$ and the canonical coordinates $\qty{q_i}$ and momenta $\qty{p_i}$ as
\begin{equation*}
	H = \sum_i p_i \dot q_i - L
\end{equation*}
\end{definition}
\begin{theorem}[Hamiltonian equations of motion]
	Let $H(\qty{q_i, p_i}, t)$ be a Hamiltonian (derived from a Lagrangian $L$, or from another Hamiltonian by a canonical transformation). Then, for all $i$,
\begin{gather*}
	\dot q_i = \dpt{H}{p_i},\\
	\dot p_i = -\dpt{H}{q_i}.
\end{gather*}
Furthermore, if $H$ is derived from a Lagrangian $L$, then
\begin{equation*}
	\frac{dH}{dt} = \dpt{H}{t} = -\dpt{L}{t}.
\end{equation*}

\end{theorem}
\begin{definition}[Canonical transformation]
	A transformation $({p_i, q_i}) \ra (P_i, Q_i)$ is \emph{canonical} if it preserves the form of the Hamiltonian equations of motion, i.e.
\begin{align*}
	\dot Q &= \dpt{H}{P} \\
	\dot P &= -\dpt{H}{Q}
\end{align*}
\end{definition}
\begin{definition}[Poisson bracket]
	The Poisson bracket of two functions $f$ and $g$, with respect to coordinates $\qty{q_i,p_i}$, is defined as
\begin{equation*}
	\{f,g\}_\qty{q_i,p_i} = \sum_i \qty(\dpt{f}{q_i} \dpt{g}{p_i} - \dpt{f}{p_i} \dpt{g}{q_i})
\end{equation*}
\end{definition}
\begin{theorem}
	A transformation is canonical iff it preserves Poisson brackets:
	\begin{equation*}
		\qty{f,g}_\qty{q_i,p_i} = \qty{f,g}_\qty{Q_i,P_i} \quad \text{for all functions $f, g$}
	\end{equation*}
\end{theorem} 

\begin{theorem}[Hamilton-Jacobi equation]
	Let $H$ be a Hamiltonian function and $S$ a generating function such that $H + \dpt{S}{t} = 0$. Then
	\begin{equation*}
		H\qty(q_1, \cdots, q_n, \dpt{S}{q_1}, \cdots, \dpt{S}{q_1}, t) + \dpt{S}{t} = 0
	\end{equation*}
	\todo{Not much of a theorem, really just by construction. But useful.}
\end{theorem}

\section{Central force problem}
\begin{theorem}[Virial theorem]
	\begin{equation*}
		\bar T = -\frac{1}{2} \overline{\sum_i\vec F_i \cdot \vec r_i}
	\end{equation*}
\end{theorem}

\section{Rotating coordinate systems}
\begin{theorem}[Effective force]
	If a particle is applied a force of $\mathbf{F}$ in a reference system, an observer in a coordinate system rotating with angular velocity $\bm \omega$ will see it as affected by a force
\begin{equation*}
	\mathbf{F}_\text{eff} = \mathbf{F} - 2m(\bm{\omega} \times \mathbf{v_r}) - m \bm{\omega} \times(\bm{\omega} \times \mathbf{r})
\end{equation*}
where $\mathbf{v_r}$ is the velocity of the particle in the rotating coordinate system.
\end{theorem}

\end{document}
