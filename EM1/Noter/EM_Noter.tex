\documentclass[]{article}
\usepackage{preamble}
\usepackage[danish]{babel}
\usepackage{amsthm}
\usepackage{bm}
\newcommand{\fr}{\mathfrak{r}}
\newcommand{\vfr}{\vec{\mathfrak{r}}}
\newcommand{\hfr}{\hat{\mathfrak{r}}}
\newcommand{\vr}{\hat{r}}
\newcommand{\hr}{\hat{\mathbf{r}}}
\newcommand{\hth}{\hat{\bm{\theta}}}
\newcommand{\hphi}{\hat{\bm{\phi}}}
\newcommand{\hn}{\hat {\mathbf{n}}}


\newtheorem{theorem}{Sætning}
\newtheorem{postulate}{Postulat}

\theoremstyle{definition}
\newtheorem{definition}{Definition}
\newtheorem{remark}{Bemærkning}
\newtheorem{experimental}{Eksperimentalt faktum}
\newcommand{\Qenc}{Q_\text{enc}}
\newcommand{\Ienc}{I_\text{enc}}

\usepackage{siunitx}

%opening
\title{Noter til EM1 (Griffiths)}

\begin{document}

\maketitle

\section{Vektoranalyse}

\section{Elektrostatik}

\begin{definition}[Elektrisk felt]
	Det elektriske felt $ \mathbf{E}(\mathbf{r})$ i et punkt $\mathbf{r}$ er den kraft pr. ladning, som en punktladning $q$ anbragt i $\mathbf{r} $ vil opleve.
\end{definition}

\begin{experimental}[Coulombs lov]
	Betragt to punktladninger $q_1$ og $q_2$, og antag, at $q_1$ er stationær. Da vil $q_1$ påvirke $q_2$ med en elektrisk kraft
	\begin{equation*}
		\mathbf{F}_E = k_e \dfrac{q_1 q_2}{\fr^2} \hfr
	\end{equation*}
	hvor $\vfr$ er vektoren fra $q_1$ til $q_2$, og $k_e = \dfrac{1}{4\pi \epsilon_0}$ er Coulombs konstant.
	
	Man kan ækvivalent sige, at $q_1$ producerer et elektrisk felt
	\begin{equation*}
		\mathbf{E} = k_e \dfrac{q_1}{\fr^2} \hfr
	\end{equation*}
	i ethvert punkt $\mathbf{r}$ i rummet.
\end{experimental}

\begin{experimental}[Superpositionsprincippet]
	Den elektriske kraft mellem to ladninger er upåvirket af alle andre ladninger i universet.
\end{experimental}


\begin{theorem}[Gauss' lov på integralform]
	Lad $\mathcal{S}$ være en flade, som ikke indeholder punkt- eller overflade-ladninger på selve fladen. Så er
	\begin{equation*}
		\oint \mathbf{E} \cdot d \mathbf{a} = \frac{\Qenc}{\epsilon_0}, 
	\end{equation*}
	hvor $\mathbf{a}$ er den udadpegende normalvektor i hvert punkt på $S$, og $\Qenc$ er den totale ladning omsluttet af $\mathcal{S}$.
\end{theorem}

\begin{theorem}[Gauss' lov på differentialform]
	I et punkt, som ikke indeholder punkt- eller overflade-ladninger, er
	\begin{equation*}
		\nabla \cdot \mathbf{E} = \dfrac{\rho}{\epsilon_0.}
	\end{equation*}
	hvor $\rho$ er ladningstætheden i punktet.
\end{theorem}

\begin{theorem}
	Det elektriske felt frembragt af en statisk ladningsfordeling er rotationsfrit:
	\begin{equation*}
		\nabla \times \mathbf{E} = \mathbf{0} \quad \text{for statiske ladningsfordelinger.}
	\end{equation*}
\end{theorem}

\begin{definition}[Elektrostatisk potentiale]
	Potentialet for et punkt i en statisk ladningsfordeling, relativt til et referencepunkt $\mathcal{O}$, er
	\begin{equation*}
		V(\mathbf{r}) = -\int_\mathcal{O}^{\mathbf{r}} \mathbf{E} \cdot d \mathbf{l}
	\end{equation*}
	Det elektriske felt er altså \emph{minus} gradienten af potentialet:
	\begin{equation*}
		\mathbf{E} = -\nabla V
	\end{equation*}
\end{definition}

\begin{theorem}[Poissons ligning]
	Det elektrostatiske potentiale opfylder 
	\begin{equation*}
		\nabla^2 V = -\frac{\rho}{\epsilon_0}.
	\end{equation*}
	For $\rho = 0$ fås $	\nabla^2 V  = 0$ (Laplaces ligning).
\end{theorem}

\begin{theorem}
	Det arbejde $W$, der kræves for at flytte en partikel over en potentialeforskel $V$, er givet ved
	\begin{equation*}
		W = qV.
	\end{equation*}
	Denne energi kan også anses som den elektrostatiske potentielle energi, som ladningen har 'vundet' ved at bevæge sig over potentialeforskellen.
\end{theorem}

\begin{definition}[Perfekt leder]
	En (perfekt) leder er et materiale med en ubegrænset mængde ladninger, der kan bevæge sig frit rundt i materialet.
\end{definition}

\begin{theorem}[Egenskaber ved ledere]
	En leder har følgende egenskaber:
	\begin{enumerate}
		\item  $\mathbf{E} = \mathbf{0}$ i det ledende materiale.
		\item  Ladningstætheden $\rho = 0$ inde i det ledende materiale.
		\item Enhver fri ladning i lederen befinder sig på overfladen.
		\item Ethvert punkt i et ledende materiale har samme eletriske potentiale.
		\item $\mathbf{E}$-feltet udenfor en leder er vinkelret på lederens overflade.
	\end{enumerate} 
\end{theorem}

\begin{theorem}
	Et punkt, der befinder sig i et hulrum inde for en leder, er elektrisk isoleret fra omgivelserne udenfor lederen.
\end{theorem}

\section{Potentialer}

\begin{theorem}[Første entydighedssætning]
	En løsning til Laplaces ligning, $\nabla^2 V = 0$, i et volumen er entydigt bestemt af værdierne af $V$ på randen af dette volumen.
	
	Dette retfærdiggør billedmetoden, hvor et elektrostatisk problem (finde et potentiale) kan løses ved at løse et helt andet fysisk problem!
\end{theorem}

\begin{theorem}[Ideel elektrisk dipol]
	En ideel elektrisk dipol i origo med dipolmoment $\mathbf{p}$, pegende i $z$-retningen, skaber et potentiale
	\begin{equation*}
		V_\text{dip}(\mathbf{r}) = \frac{1}{4\pi \epsilon_0} \frac{\mathbf{p} \cdot \hat {\mathbf{r}}}{r^2}
	\end{equation*}
	og et elektrisk felt
	\begin{align*}
		\mathbf{E}_\text{dip}(r, \theta) &= \frac{p}{4\pi \epsilon_0  r^3}(2 \cos \theta \hr + \sin \theta \hth)\\
		&= \frac{1}{4\pi \epsilon_0  r^3}(3(\mathbf{p}\times \hr) - \mathbf{p}).
	\end{align*}
\end{theorem}

\section{Elektriske felter i materie}

\begin{theorem}[Bundne ladninger]
	En polarisation $\mathbf{P}$ skaber en bunden overfladeladning
	\begin{equation*}
		\sigma_b = \mathbf{P} \cdot \hat {\mathbf{n}} 
	\end{equation*}
	og en bunden volumenladning
	\begin{equation*}
		\rho_b = - \nabla \cdot \mathbf{P}.
	\end{equation*}
\end{theorem}

\begin{definition}[Elektrisk forskydning]
	Den elektriske forskydning $\mathbf{D}$ defineres ved
	%	Felterne $\mathbf{E}$, $\mathbf{D}$ og $\mathbf{P}$ er relateret ved
	\begin{equation*}
		\mathbf{D} = \epsilon_0 \mathbf{E}+\mathbf{P}.
	\end{equation*}
\end{definition}

\begin{theorem}[Gauss' lov for $\mathbf{D}$]
	Gauss' lov gælder for $D$, men kun med de frie ladninger:
	\begin{equation*}
		\nabla \cdot \mathbf{D} = \rho_f
	\end{equation*}
	eller på integralform:
	\begin{equation*}
		\oint \mathbf{D} \cdot d\mathbf{a} = Q_{f, enc}
	\end{equation*}
\end{theorem}

\section{Magnetostatik}
\begin{experimental}[Magnetisk kraft]
	En ladning $q$, der bevæger sig med hastighed $v$ gennem et konstant magnetfelt $\mathbf{B}$, vil opleve en magnetisk kraft
	\begin{equation*}
		\mathbf{F}_\text{mag} = q(\mathbf{v} \times \mathbf{B}).
	\end{equation*}
\end{experimental}
\begin{theorem}
	Magnetiske felter kan ikke udføre arbejde (da $\mathbf{F} \perp \mathbf{B}$)
\end{theorem}

\begin{theorem}[Magnetisk kraft på linje]
	Den magnetiske kraft på et stykke strømførende ledning er
	\begin{equation*}
		\mathbf{F}_\text{mag} = \int(\mathbf{I} \times \mathbf{B}) \ dl = \int I (d\mathbf{l} \times \mathbf{B}).
	\end{equation*}
\end{theorem}

\begin{definition}[Overfladestrøm]
	Når en strøm bevæger sig over en overflade, beskrives det ved en overfladestrøm
	\begin{equation*}
		\mathbf{K} = \dfrac{d\mathbf{I}}{d_{l_\perp}}
	\end{equation*}
	hvor $d_{l_\perp}$ er den infitisemale bredde af et bånd tangentielt til strømmens retning.
\end{definition}

\begin{definition}[Volumenstrøm]
	Når en strøm bevæger sig gennem en volumen, beskrives det ved en volumenstrøm
	\begin{equation*}
		\mathbf{J} = \dfrac{d\mathbf{I}}{d_{a_\perp}}
	\end{equation*}
	hvor $d_{a_\perp}$ er det infitisemale tværsnitsareal af et rør tangentielt på strømmen.
\end{definition}

\begin{theorem}[Kontinuitetssætningen]
	Ladning er lokalt bevaret. Matematisk set hænger volumenstrøm og ladningstæthed derfor sammen ved
	\begin{equation*}
		\nabla \cdot \mathbf{J} = -\dpt{\rho}{t}.
	\end{equation*}
\end{theorem}

\begin{experimental}[Biot-Savarts lov]
	Magnetfeltet i et punkt $\mathbf{r}$ i rummet produceret af en jævn linjestrøm $\mathbf{I}(\textbf{r'})$ er givet ved
	\begin{equation*}
		\mathbf{B}(\mathbf{r}) = \frac{\mu_0}{4 \pi} \int{\frac{\mathbf{I} \times \hfr}{\fr^2}} \ dl.
	\end{equation*}
	hvor $dl$ løber over hele den kurve, strømmen løber langs.
\end{experimental}

\begin{experimental}
	Der er hidtil ikke fundet magnetiske monopoler. Matematisk set er altså, så vidt vi har set,
	\begin{equation*}
		\nabla \times \mathbf{B} = \mathbf{0}.
	\end{equation*}
\end{experimental}

\begin{theorem}[Ampères lov på integralform]
	Lad $\gamma$ være en lukket kurve uden linje- eller overfladestrøm på selve kurven og kun jævn volumenstrøm. Så er
	\begin{equation*}
		\oint_\gamma \mathbf{B} \cdot d \mathbf{l} = \mu_0 \Ienc, 
	\end{equation*}
	hvor $\Ienc$ er den totale ladning omsluttet af en flade, der har $\gamma$ som rand.
\end{theorem}

\begin{theorem}[Ampères lov på differentialform]
	I et punkt, som ikke gennemløbes af linje- eller overfladestrømme, og hvor volumenstrømmen er jævn, er 
	\begin{equation*}
		\nabla \times \mathbf{B} = \mu_0 \mathbf{J}.
	\end{equation*}
	hvor $\mathbf{J}$ er volumenstrømmen i punktet.
\end{theorem}

\begin{definition}[Magnetisk vektorpotentiale]
	Da $\mathbf{B}$-feltet er divergensfrit, kan det udtrykkes som rotationen af en vektor $\mathbf{A}$:
	\begin{equation*}
		B = \nabla \times \mathbf{A}
	\end{equation*}
	Man vælger altid at lægge en konstant til $\mathbf{A}$, således at
	\begin{equation*}
		\nabla \cdot \mathbf{A} = 0.
	\end{equation*}
\end{definition}

\begin{theorem}[Ampéres lov for vektorpotentialet]
	Ampères lov kan udtrykkes som en Poisson-sammenhæng mellem vektorpotentialet og volumenstrømmen:
	\begin{equation*}
		\nabla^2 \mathbf{A} = - \mu_0 \mathbf{J}
	\end{equation*}
	Der er tre ligninger her, en for hver \textbf{kartesiske} komponent af $\mathbf{A}$ og $\mathbf{J}$!
	
	Hvis $ \mathbf{J} \rightarrow 0 $ i det uendelige, er løsningen 
	
	\begin{equation*}
		\mathbf{A}(\mathbf{r}) = \frac{\mu_0}{4 \pi } \int \dfrac{\mathbf{J}(\mathbf{r'})}{\fr} d \tau'.
	\end{equation*}
\end{theorem}

\begin{definition}[Magnetisk dipolmoment]
	Det magnetiske dipolmoment af en løkke, der bærer en strøm $I$, er defineret som
	\begin{equation*}
		\mathbf{m} = I \int_\mathcal{S}{d\mathbf{a}}
	\end{equation*}
	for enhver flade $\mathcal{S}$, der har strømløkken som rand. Hvis $\mathcal{S}$ er flad (dvs. ligger i en plan), er $\abs{\int_\mathcal{S}{d\mathbf{a}}} = A(\mathcal{S})$, mens retningen kommer fra højrehåndsreglen.
\end{definition}

\begin{theorem}[Ideel magnetisk dipol]
	En ideel magnetisk dipol i origo med dipolmoment $\mathbf{m}$, pegende i $z$-retningen, skaber et vektorpotentiale
	\begin{equation*}
		\mathbf{A}_\text{dip}(\mathbf{r}) = \frac{\mu_0}{4\pi} \frac{\mathbf{m} \times \hat {\mathbf{r}}}{r^2}
	\end{equation*}
	og et magnetisk felt
	\begin{align*}
		\mathbf{B}_\text{dip}(\mathbf{r}) &= \frac{\mu_0m}{4\pi r^3}(2 \cos \theta \hr + \sin \theta \hth)\\
		&= \frac{\mu_0}{4\pi r^3}(3(\mathbf{m}\times \hr) - \mathbf{m}).
	\end{align*}
	Dette er præcis de samme ligninger som som en elektrisk dipol -- eneste ændringer er $\frac{1}{\epsilon_0} \ra \mu_0$, $p \ra \mathbf{m} $ og $p \cdot \hr \rightarrow \mathbf{m}\times \hr $!
\end{theorem}

\section{Magnetiske felter i materie}
%\begin{theorem}
%	Felterne $\mathbf{B}$, $\mathbf{H}$ og $\mathbf{M}$ er relateret ved
%	\begin{equation*}
%		\mathbf{B} = \mu_0(\mathbf{H}+\mathbf{M})
%	\end{equation*}
%\end{theorem}

\begin{definition}[H-feltet]
	\begin{equation*}
		\mathbf{H} = \dfrac{1}{\mu_0} \mathbf{B} + \mathbf{M}
	\end{equation*}
\end{definition}

\begin{theorem}[Amperes lov for $H$]
	Der gælder:
	\begin{equation*}
		\nabla \times \mathbf{B} = \mathbf{J}_{f, \text{enc}}
	\end{equation*}
	eller på integralform:
	\begin{equation*}
		\oint \mathbf{H} \cdot d \mathbf{l} = I_{f, \text{enc}}
	\end{equation*}
\end{theorem}

\begin{theorem}[Bundne strømme]
	En magnetisering $\mathbf{M}$ skaber en bunden overfladestrøm
	\begin{equation*}
		\mathbf{K}_b = \mathbf{M} \times \hn
	\end{equation*}
	og en bunden volumenladning
	\begin{equation*}
		\mathbf{J}_b = \nabla \times \mathbf{M}.
	\end{equation*}
\end{theorem}

\section{Elektrodynamik}

\begin{theorem}[Modstand i en ledning]
	Den totale modstand $R$ i en ledning med længde $\ell$, tværsnitsareal $A$ og resistivitet $\rho$, er givet ved
	\begin{equation*}
		R = \frac{L}{A}\rho.
	\end{equation*}
\end{theorem}

\begin{definition}[Elektromotans]
	\emph{Elektromotansen} $\mathcal{E}$ (ofte, misvisende, kaldet for den \emph{elektromotoriske kraft}) er integralet af de kræfter, der skubber ladninger rundt i en strømløkke:
	\begin{equation*}
		\mathcal{E} := \oint \mathbf{f} \cdot d\mathbf{l} = \oint \mathbf{f_s} \cdot d\mathbf{l}.
	\end{equation*}
	da $\mathbf{f} = \mathbf{f}_s + \mathbf{E}$. For en jævn strøm giver $\mathbf{E}$ en elektrostatisk (dvs. rotationsfri) kraft, så  $\oint \mathbf{E} \cdot d\mathbf{l} = 0$.
\end{definition}

\begin{remark}
	$\mathcal{E}$ kan oftest også betragtes som arbejde pr. ladning i kredsløbet, og i en ideel elektromotorisk kilde er $\mathcal{E} = V$.
\end{remark}

\begin{theorem}[Fluxregel for elektromotans]
	I næsten alle situationer, hvor der induceres en strøm, er
\begin{equation*} 
	\mathcal{E} = \oint{\mathbf{E} \cdot d\mathbf{l}} = -\dfrac{d \Phi_B}{dt}.
\end{equation*}
\end{theorem}

\begin{theorem}[Lenz' lov]
	Den inducerede strøm løber i den retning, der mindsker ændringen i magnetisk flux.
\end{theorem}

\begin{definition}[Gensidig induktans]
	Den gensidige induktans $M$ mellem to løkker er en proportionalitetskonstant for den magnetiske flux, der induceres i den ene, hvis der løber en strøm gennem den anden:
	\begin{equation*}
		\Phi_{B,1} = M \cdot I_2.
	\end{equation*}
	(Neumann-formlen, udledt vha. Stokes og vektorpotentialet, viser, at den gensidige induktans er symmetrisk: $ \dfrac{\Phi_1}{I_2} = \dfrac{\Phi_2}{I_1}$)
	\begin{theorem}
		I samme situation er
		\begin{equation*}
			\mathcal{E}_2 = -\frac{d \Phi_{B,2}}{dt} = - M \frac{d I}{dt}.
		\end{equation*}
	\end{theorem}
\end{definition}

\begin{theorem}[R/L-kredsløb]
	I et kredsløb med induktans $L$ og modstand $R$, forbundet til et batteri, der leverer emf $\mathcal{E}_0$, gælder differentialligningen
	\begin{equation*}
		\mathcal{E}_0 - L\frac{d I}{dt} = IR,
	\end{equation*}
	med løsningen
	\begin{equation*}
		I(t) = \dfrac{\mathcal{E}_0}{R} + k \exp(-\frac{Rt}{L}).
	\end{equation*}
\end{theorem}

\begin{definition}[Selvinduktans]
	\emph{(Selv)induktansen} $L$ af en løkke fortæller om den magnetiske flux, den inducerer i sig selv ved en strømændring:
	\begin{equation*}
		\Phi_B = LI
	\end{equation*}
\end{definition}

\begin{theorem}[Magnetisk energi]
	Et kredsløb med en induktans $L$, hvor der løber en strøm $I$, bærer en magnetisk energi
	\begin{equation*}
		W = \dfrac{1}{2} LI^2.
	\end{equation*}
\end{theorem}

\begin{theorem}[Ampères lov med forskydningsled]
	Hvis strømmen ændrer sig, skal Ampers lov have et ekstra led:,
	\begin{equation*}
		\nabla \times \mathbf{B} = \mu_0 \mathbf{J} + \mu_0 \epsilon_0 \dpt{\mathbf{E}}{t}.
	\end{equation*}
	Størrelsen $ \epsilon_0 \dpt{\mathbf{E}}{t}$ kaldes traditionelt (og misvisende) \emph{forskydningsstrømmen}.
\end{theorem}

%\begin{environment-name}
%	content
%\end{environment-name}

\newpage
\section{Enheder}
Elektrisk ladning $q$:
\begin{equation*}
	[q] = \unit{C} = \unit{A \cdot s}
\end{equation*}
Elektrisk strøm $I$:
\begin{equation*}
	[I] = \unit{A} = \unit{\frac{C}{s}}
\end{equation*}
Overfladestrøm $K$:
\begin{equation*}
	[K] = \unit{\frac{A}{m}}
\end{equation*}
Volumenstrøm $J$:
\begin{equation*}
	[J] = \unit{\frac{A}{m^2}}
\end{equation*}
Elektrisk potentiale $V$:
\begin{equation*}
	[V] = \unit{V} = \unit{\frac{J}{C}} = \unit{\frac{N \cdot m}{C}} = \unit{\frac{N \cdot m}{A \cdot s}}
\end{equation*}
Elektrisk felt $E$:
\begin{equation*}
	[E] = \unit{\frac{N}{C}} = \unit{\frac{V}{m}} = \unit{\frac{m\cdot kg}{s^3 \cdot A}}
\end{equation*}
Elektrisk forskydning $D$ og polarisering $P$:
\begin{equation*}
	[D] = [P] = [\epsilon_0 E] = \unit{\frac{C}{m^2}} = \unit{\frac{A \cdot s}{m^2}}
\end{equation*}
Magnetfelt $B$:
\begin{equation*}
	[B] = \unit{T} = \unit{\frac{N}{A \cdot m} =  \unit{\frac{N \cdot s}{C \cdot m}} = \unit{\frac{kg}{A \cdot s^2}}}
\end{equation*}
$H$-felt og magnetisering $M$:
\begin{equation*}
	[H] = [M] = \qty[\frac{B}{\mu_0}] = \unit{\frac{A}{m}}
\end{equation*}
Magnetisk flux $ \Phi_B$:
\begin{equation*}
	[\Phi_B] = \unit{Wb} = \unit{T \cdot m^2} = \unit{V \cdot s}
\end{equation*}
Magnetisk dipolmoment $m$:
\begin{equation*}
	[m] = [Ia] = \unit{A \cdot m^2}
\end{equation*}
Modstand (resistans) $R$:
\begin{equation*}
	[R] = \unit{\Omega} = \unit{\frac{V}{A}} = \unit{\dfrac{J \cdot s}{C^2}}
\end{equation*}
Resistivitet $\rho$:
\begin{equation*}
	[\rho] = \unit{\Omega \cdot m} = \unit{\frac{V \cdot m}{A}}
\end{equation*}
Induktans $L$:
\begin{equation*}
	[L] = [H] = \unit{\frac{V\cdot s}{A}} = \unit{\frac{J}{A^2}}
\end{equation*}

\section{Konstanter}
Vakuumpermittivitet:
\begin{equation*}
	\epsilon_0 = \SI{8.854187813e-12}{\dfrac{C^2}{N \cdot m^2}}
\end{equation*}
Coulomb-konstant:
\begin{equation*}
	k_e = \frac{1}{4\pi \epsilon_0} = \SI{8.98755179e9}{\dfrac{N \cdot m^2}{C^2}}
\end{equation*}
Vakuumpermeabilitet:
\begin{equation*}
	\mu_0 = 4\pi \cdot \SI{e-7}{\frac{N}{A^2}} = \SI{1.256637062e-6}{\frac{N}{A^2}}
\end{equation*}
Lysets fart:
\begin{equation*}
	c = \SI{2.99 792 458e9}{\frac{m}{s}}
\end{equation*}
Der gælder:
\begin{equation*}
	\epsilon_0 \mu_0 = \dfrac{1}{c^2}
\end{equation*}
Elementarladning:
\begin{equation*}
	e = \SI{1.602176634e-19}{C}
\end{equation*}
Elektronens masse:
\begin{equation*}
	m_e = \SI{9.109383701e-31}{kg}
\end{equation*}
Protonens masse:
\begin{equation*}
	m_p = \SI{1.6726219237e-27}{kg}
\end{equation*}

%\newpage
\end{document}
