\documentclass[]{article}
\usepackage{preamble}
\usepackage[danish]{babel}
\usepackage{amsthm}
\usepackage{bm}
\newcommand{\fr}{\mathfrak{r}}
\newcommand{\vfr}{\vec{\mathfrak{r}}}
\newcommand{\hfr}{\hat{\mathfrak{r}}}
\newcommand{\vr}{\hat{r}}
\newcommand{\hr}{\hat{\mathbf{r}}}
\newcommand{\hth}{\hat{\bm{\theta}}}
\newcommand{\hphi}{\hat{\bm{\phi}}}

\newtheorem{theorem}{Sætning}
\newtheorem{postulate}{Postulat}

\theoremstyle{definition}
\newtheorem{definition}{Definition}
\newtheorem{remark}{Bemærkning}
\newtheorem{experimental}{Eksperimentalt faktum}
\newcommand{\Qenc}{Q_\text{enc}}
\newcommand{\Ienc}{I_\text{enc}}

\usepackage{siunitx}

%opening
\title{Noter til EM1}

\begin{document}

\maketitle

\section{Elektrostatik}

\begin{definition}[Elektrisk felt]
	Det elektriske felt $ \mathbf{E}(\mathbf{r})$ i et punkt $\mathbf{r}$ er den kraft pr. ladning, som en punktladning $q$ anbragt i $\mathbf{r} $ vil opleve.
\end{definition}

\begin{experimental}[Coulombs lov]
	Betragt to punktladninger $q_1$ og $q_2$, og antag, at $q_1$ er stationær. Da vil $q_1$ påvirke $q_2$ med en elektrisk kraft
	\begin{equation*}
		\mathbf{F}_E = k_e \dfrac{q_1 q_2}{\fr^2} \hfr
	\end{equation*}
	hvor $\vfr$ er vektoren fra $q_1$ til $q_2$, og $k_e = \dfrac{1}{4\pi \epsilon_0}$ er Coulombs konstant.
	
	Man kan ækvivalent sige, at $q_1$ producerer et elektrisk felt
	\begin{equation*}
		\mathbf{E} = k_e \dfrac{q_1}{\fr^2} \hfr
	\end{equation*}
	i ethvert punkt $\mathbf{r}$ i rummet.
\end{experimental}

\begin{experimental}[Superpositionsprincippet]
	Den elektriske kraft mellem to ladninger er upåvirket af alle andre ladninger i universet.
\end{experimental}


\begin{theorem}[Gauss' lov på integralform]
	Lad $\mathcal{S}$ være en flade, som ikke indeholder punkt- eller overflade-ladninger på selve fladen. Så er
	\begin{equation*}
		\oint \mathbf{E} \cdot d \mathbf{a} = \frac{\Qenc}{\epsilon_0}, 
	\end{equation*}
	hvor $\mathbf{a}$ er den udadpegende normalvektor i hvert punkt på $S$, og $\Qenc$ er den totale ladning omsluttet af $\mathcal{S}$.
\end{theorem}

\begin{theorem}[Gauss' lov på differentialform]
	I et punkt, som ikke indeholder punkt- eller overflade-ladninger, er
	\begin{equation*}
		\nabla \cdot \mathbf{E} = \dfrac{\rho}{\epsilon_0.}
	\end{equation*}
	hvor $\rho$ er ladningstætheden i punktet.
\end{theorem}

\begin{theorem}
	Det elektriske felt frembragt af en statisk ladningsfordeling er rotationsfrit:
	\begin{equation*}
		\nabla \times \mathbf{E} = \mathbf{0} \quad \text{for statiske ladningsfordelinger.}
	\end{equation*}
\end{theorem}

\begin{definition}[Elektrostatisk potentiale]
	Potentialet for et punkt i en statisk ladningsfordeling, relativt til et referencepunkt $\mathcal{O}$, er
	\begin{equation*}
		V(\mathbf{r}) = -\int_\mathcal{O}^{\mathbf{r}} \mathbf{E} \cdot d \mathbf{l}
	\end{equation*}
	Det elektriske felt er altså \emph{minus} gradienten af potentialet:
	\begin{equation*}
		\mathbf{E} = -\nabla V
	\end{equation*}
\end{definition}

\begin{theorem}
	Det arbejde $W$, der kræves for at flytte en partikel over en potentialeforskel $V$, er givet ved
	\begin{equation*}
		W = qV.
	\end{equation*}
	Denne energi kan også anses som den elektrostatiske potentielle energi, som ladningen har 'vundet' ved at bevæge sig over potentialeforskellen.
\end{theorem}

\begin{definition}[Perfekt leder]
	En (perfekt) leder er et materiale med en ubegrænset mængde ladninger, der kan bevæge sig frit rundt i materialet.
\end{definition}

\begin{theorem}[Egenskaber ved ledere]
	En leder har følgende egenskaber:
	\begin{enumerate}
		\item  $\mathbf{E} = \mathbf{0}$ i det ledende materiale.
		\item  Ladningstætheden $\rho = 0$ inde i det ledende materiale.
		\item Enhver fri ladning i lederen befinder sig på overfladen.
		\item Ethvert punkt i et ledende materiale har samme eletriske potentiale.
		\item $\mathbf{E}$-feltet udenfor en leder er vinkelret på lederens overflade.
	\end{enumerate} 
\end{theorem}

\begin{theorem}
	Et punkt, der befinder sig i et hulrum inde for en leder, er elektrisk isoleret fra omgivelserne udenfor lederen.
\end{theorem}

\section{Magnetostatik}
\begin{experimental}[Magnetisk kraft]
	En ladning $q$, der bevæger sig med hastighed $v$ gennem et konstant magnetfelt $\mathbf{B}$, vil opleve en magnetisk kraft
	\begin{equation*}
		\mathbf{F}_\text{mag} = q(\mathbf{v} \times \mathbf{B}).
	\end{equation*}
\end{experimental}
\begin{theorem}
	Magnetiske felter kan ikke udføre arbejde (da $\mathbf{F} \perp \mathbf{B}$)
\end{theorem}

\begin{experimental}[Biot-Savarts lov]
	Magnetfeltet i et punkt $\mathbf{r}$ i rummet produceret af en jævn linjestrøm $\mathbf{I}(\textbf{r'})$ er givet ved
	\begin{equation*}
		\mathbf{B}(\mathbf{r}) = \frac{\mu_0}{4 \pi} \int{\frac{\mathbf{I} \times \hfr}{\fr^2}} \ dl.
	\end{equation*}
	hvor $dl$ løber over hele den kurve, strømmen løber langs.
\end{experimental}

\begin{experimental}
	Der er hidtil ikke fundet magnetiske monopoler. Matematisk set er altså, så vidt vi har set,
	\begin{equation*}
		\nabla \times \mathbf{B} = \mathbf{0}.
	\end{equation*}
\end{experimental}

\begin{theorem}[Amperes lov på integralform]
	Lad $\gamma$ være en lukket kurve uden linje- eller overfladestrøm på selve kurven. Så er
	\begin{equation*}
		\oint_\gamma \mathbf{B} \cdot d \mathbf{l} = \mu_0 \Ienc, 
	\end{equation*}
	hvor $\Ienc$ er den totale ladning omsluttet af en flade, der har $\gamma$ som rand.
\end{theorem}

\begin{theorem}[Amperes lov på differentialform]
	I et punkt, som ikke gennemløbes af linje- eller overfladestrømme, er
	\begin{equation*}
		\nabla \times \mathbf{B} = \mu_0 \mathbf{J}.
	\end{equation*}
	hvor $J$ er volumenstrømmen i punktet.
\end{theorem}

\begin{definition}[Magnetisk dipolmoment]
	Det magnetiske dipolmoment af en løkke, der bærer en strøm $I$, er defineret som
	\begin{equation*}
		\mathbf{m} = I \int_\mathcal{S}{d\mathbf{a}}
	\end{equation*}
	for enhver flade $\mathcal{S}$, der har strømløkken som rand. Hvis $\mathcal{S}$ er flad (dvs. ligger i en plan), er $\abs{\int_\mathcal{S}{d\mathbf{a}}} = A(\mathcal{S})$, mens retningen kommer fra højrehåndsreglen.
\end{definition}

\begin{theorem}
	En magnetisk dipol i origo med dipolmoment $\mathbf{m}$, pegende i $z$-retningen, skaber et vektorpotentiale
	\begin{equation*}
		\mathbf{A}_\text{dip}(\mathbf{r}) = \frac{\mu_0}{4\pi} \frac{\mathbf{m} \times \hat {\mathbf{r}}}{r^2}
	\end{equation*}
	og et magnetisk felt
	\begin{align*}
		\mathbf{B}_\text{dip}(\mathbf{r}) &= \frac{\mu_0m}{4\pi r^3}(2 \cos \theta \hr + \sin \theta \hth)\\
		&= \frac{\mu_0}{4\pi r^3}(3(\mathbf{m}\times \hr) - \mathbf{m}).
	\end{align*}
\end{theorem}

\section{Elektrodynamik}

\begin{definition}[Elektromotans]
	\emph{Elektromotansen} (ofte, misvisende, kaldet for den \emph{elektromotoriske kraft}) er integralet af de kræfter, der skubber ladninger rundt i en strømløkke:
	\begin{equation*}
		\mathcal{E} := \oint \mathbf{f} \cdot d\mathbf{l} = \oint \mathbf{f_s} \cdot d\mathbf{l}.
	\end{equation*}
	da $\mathbf{f} = \mathbf{f}_s + \mathbf{E}$. For en jævn strøm giver $\mathbf{E}$ en elektrostatisk (dvs. rotationsfri) kraft, så  $\oint \mathbf{E} \cdot d\mathbf{l} = 0$.
\end{definition}

\begin{remark}
	$\mathcal{E}$ kan oftest også betragtes som arbejde pr. ladning i kredsløbet, og i en ideel elektromotorisk kilde er $\mathcal{E} = V$.
\end{remark}

\begin{theorem}
	I næsten alle situationer er
\begin{equation*} 
	\mathcal{E} = -\dfrac{d \Phi_B}{dt}
\end{equation*}
\end{theorem}

\begin{theorem}[Lenz' lov]
	Den inducerede strøm løber i den retning, der mindsker ændringen i magnetisk flux.
\end{theorem}

\begin{definition}[Gensidig induktans]
	Den gensidige induktans $M$ mellem to løkker er en proportionalitetskonstant for den flux, der induceres i den ene, hvis der løber en strøm gennem den anden:
	\begin{equation*}
		\phi_1 = MI_2.
	\end{equation*}
	(Neumann-formlen, udledt vha. Stokes og vektorpotentialet, viser, at den gensidige induktans er symmetrisk: $ \dfrac{\Phi_1}{I_2} = \dfrac{\Phi_2}{I_1}$)
	Vi får også:
	\begin{theorem}
		I samme situation er
		\begin{equation*}
			\mathcal{E}_2 = -\frac{d \Phi_2}{dt} = - M \dot I
		\end{equation*}
	\end{theorem}
\end{definition}

\begin{definition}[Selvinduktans]
	\emph{(Selv)induktansen} $L$ af en løkke er den flux, den inducerer i sig selv ved en strømændring:
	\begin{equation*}
		\phi 
	\end{equation*}
\end{definition}

%\begin{environment-name}
%	content
%\end{environment-name}

\section{Konstanter}
Vakuumpermittivitet:
\begin{equation*}
	\epsilon_0 = \SI{8.854187813e-12}{\dfrac{C^2}{N \cdot m^2}}
\end{equation*}
Coulomb-konstant:
\begin{equation*}
	k_e = \frac{1}{4\pi \epsilon_0} = \SI{8.98755179e9}{\dfrac{N \cdot m^2}{C^2}}
\end{equation*}
Vakuumpermeabilitet:
\begin{equation*}
	\mu_0 = 4\pi \cdot \SI{e-7}{\frac{N}{A^2}} = \SI{1.256637062e-6}{\frac{N}{A^2}}
\end{equation*}
Lysets fart:
\begin{equation*}
	c = \SI{2.99 792 458e9}{\frac{m}{s}}
\end{equation*}
Der gælder:
\begin{equation*}
	\epsilon_0 \mu_0 = \dfrac{1}{c^2}
\end{equation*}
Elementarladning:
\begin{equation*}
	e = \SI{1.602176634e-19}{C}
\end{equation*}
Elektronens masse:
\begin{equation*}
	m_e = \SI{9.109383701e-31}{kg}
\end{equation*}
Protonens masse:
\begin{equation*}
	m_p = \SI{1.6726219237e-27}{kg}
\end{equation*}


\section{Enheder}
Elektrisk ladning $q$:
\begin{equation*}
	[q] = \unit{C} = \unit{A \cdot s}
\end{equation*}
Elektrisk strøm $I$:
\begin{equation*}
	[I] = \unit{A} = \unit{\frac{C}{s}}
\end{equation*}
Overfladestrøm $K$:
\begin{equation*}
	[K] = \unit{\frac{A}{m}}
\end{equation*}
Volumenstrøm $J$:
\begin{equation*}
	[J] = \unit{\frac{A}{m^2}}
\end{equation*}
Elektrisk potentiale $V$:
\begin{equation*}
	[V] = \unit{V} = \unit{\frac{J}{C}} = \unit{\frac{N \cdot m}{C}}
\end{equation*}
Elektrisk felt $E$:
\begin{equation*}
	[E] = \unit{\frac{N}{C}} = \unit{\frac{V}{m}} = \unit{\frac{m\cdot kg}{s^3 \cdot A}}
\end{equation*}
Elektrisk forskydning $D$ og polarisering $P$:
\begin{equation*}
	[D] = [P] [\epsilon_0 E] \unit{\frac{C}{m^2}} = \unit{\frac{A \cdot s}{m^2}}
\end{equation*}
Magnetfelt $B$:
\begin{equation*}
	[B] = \unit{T} = \unit{\frac{N}{A \cdot m} =  \unit{\frac{N \cdot s}{C \cdot m}} = \unit{\frac{kg}{A \cdot s^2}}}
\end{equation*}
$H$-felt og magnetisering $M$:
\begin{equation*}
	[H] = [M] = \qty[\frac{B}{\mu_0}] = \unit{\frac{A}{m}}
\end{equation*}
Magnetisk flux $ \Phi_B$:
\begin{equation*}
	[\Phi_B] = \unit{Wb} = \unit{T \cdot m^2} = \unit{V \cdot s}
\end{equation*}
Magnetisk dipolmoment $m$:
\begin{equation*}
	[m] = [Ia] = \unit{A \cdot m^2}
\end{equation*}
Modstand (resistans) $R$:
\begin{equation*}
	[R] = \unit{\Omega} = \unit{\frac{V}{A}} = \unit{\dfrac{J \cdot s}{C^2}}
\end{equation*}
Resistivitet $\rho$:
\begin{equation*}
	[\rho] = \unit{\Omega \cdot m}
\end{equation*}
Induktans $L$:
\begin{equation*}
	[L] = [H] = \unit{\frac{V\cdot s}{A}} = \unit{\frac{J}{A^2}}
\end{equation*}
\end{document}
